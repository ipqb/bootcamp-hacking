\begin{frame}{Open Source Software}

 \begin{itemize}

  \item Open source software isn't just software you don't have to pay for.  
   It's software that anyone has the right to browse, modify, and distribute. 

  \item When you write a piece of software, you should give it a license so 
   that other people know whether or not they can use it.  It's usually not 
   hard to \href{http://choosealicense.com/}{choose a license} you 
   like\footnote{License summaries paraphrased from 
   \url{http://choosealicense.com}}:

  \begin{itemize}

   \item MIT License: Very few restrictions on how your code can be used.

   \item Apache License: Similar to MIT, but gives more protection to your 
    users if you're planning to patent your code.

   \item GPL License: Requires others who use your code to also use an open 
    source license.

   \end{itemize}
   
  \item Legal disclaimer: You should probably check with your PI before 
   choosing a license.

 \end{itemize}

\end{frame}
