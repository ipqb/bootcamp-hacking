\begin{frame}{Programming languages}
  \begin{itemize}
  \item Python
    \begin{itemize}
    \item ``Scientists often need to improvise when trying to interpret results, so they are drawn to dynamic languages which allow them to work very quickly and see results almost immediately'' --- Python creator Guido von Rossum
    \end{itemize}
    \pause
  \item C/C++
    \begin{itemize}
    \item Fast - use when runtime considerations
      become more important than time to implement code (rare), otherwise
      Python should be faster.
    \end{itemize}
    \pause
  \item Web programming
    \begin{itemize}
    \item Makes your software/results accessible to the world
    \end{itemize}
  \end{itemize}
\end{frame}

\begin{frame}{DNA reverse complement algorithm - Python}
  \tiny
  \inputminted{python}{code-snippets/reverse-complement.py}
\end{frame}

\begin{frame}{DNA reverse complement algorithm - C++}
  \begin{columns}[T]
    \begin{column}{0.2\textwidth}
      \fontsize{0.25mm}{0.25mm}\selectfont
      \inputminted{c++}{code-snippets/reverse-complement.cc}
    \end{column}
    \begin{column}{0.75\textwidth}
      \begin{itemize}
      \item C++ --- 280 lines
      \item Python --- 28 lines
      \item C++ program runs 5x faster, but took 10x longer to write
        (assuming that writing lines of code in various languages take
        the same amount of time, which is a decent approximation)
      \item This general trend (Python is faster to write, but slower to run)
        holds for many algorithms
      \end{itemize}
    \end{column}
  \end{columns}
\end{frame}