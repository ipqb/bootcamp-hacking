\begin{frame}{Creating a repository}

 There are two ways to create a repository:

 \begin{itemize}

  \item \texttt{git init} -- Turns the current directory into a git repository.  
   Do this when you start a new project.

  \item \texttt{git clone} -- Copy a repository onto your computer from 
   somewhere else.

  \item \href{https://github.com/}{GitHub} is a popular website that hosts 
   public git repositories for free.

 \end{itemize}

\end{frame}

\begin{frame}{Committing Your Changes}

 \begin{itemize}

  \item You have to tell git which changes you want it to keep track of.

  \item This process is called committing, and in git it has two steps:

  \item \texttt{git add} -- Add files to a staging area.

  \item \texttt{git commit} -- Commit all the files in the staging area.

 \end{itemize}
 
\end{frame}

\begin{frame}{Sharing Your Code}

 \begin{itemize}

  \item Git makes it easy to share your code and to develop it in concert with 
   other people.

  \item \texttt{git push} -- Push all of your commits to a central server.

  \item \texttt{git pull} -- Pull any commits made by others into your 
   repository.

 \end{itemize}

\end{frame}
