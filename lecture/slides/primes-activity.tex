\begin{frame}{Activity: Comparing Prime-Finding Algorithms}

 \begin{itemize}
   
  \item This activity is meant to give some some experience with git and a 
   little more practice with python.

  \item I've written two algorithms to find all the prime numbers below some 
  threshold.  The code is in \texttt{find\_primes.py}.

  \item Spend a few minutes reading the two algorithms, understanding how they 
   work, and thinking about which is more efficient and why.

 \end{itemize}

\end{frame}

\begin{frame}{Optimizing Code}

 \begin{itemize}

  \item Optimizing code is all about cutting out redundant work being done by 
   the computer.

  \item Don't optimize code unless you have to.  There's no point optimizing 
   code that isn't a bottleneck.  Doing so will probably just make your code 
   harder to understand.

  \item In addition to thinking about efficient use of the CPU, we can also 
   think about efficient use of memory.

 \end{itemize}

\end{frame}

\begin{frame}{Activity: Comparing Prime-Finding Algorithms}

 Work on the following activity for the next 15-30 minutes:

 \begin{itemize}

  \item Make a plot comparing how long each function takes as the input limit 
   gets higher and higher (i.e. as more and more primes are found).

  \begin{itemize}

   \item Put your code in \texttt{time\_primes.py}.

   \item Use \texttt{matplotlib} to create the plot.

  \end{itemize}

  \item The trial division function could be made more efficient.  If you can 
   think of an optimization, implement it and see how much faster it makes the 
   algorithm.

  \item Commit your work at least once!  My rule of thumb is to commit every 
   time I get something new working.

 \end{itemize}

\end{frame}

