\begin{frame}{BFF: The Command Line And You}
  Why use the command line?
  \begin{itemize}
  \item Using the command line is often faster than using an equivalent GUI
    (\textbf{G}raphical \textbf{U}ser \textbf{I}nterface) tool
  \item Many computing operations are only possible via the command line
  \item The command line is used to access and control remote servers,
    including scientific computing clusters
  \end{itemize}
\end{frame}

\begin{frame}{The terminal}
  \begin{itemize}
  \item The application you use to interact with the command line is called a
    \textit{terminal}
  \item On OSX, the terminal can be accessed by going to
    Applications $\rightarrow$ Utilities $\rightarrow$ Terminal
  \item Fire one up now!
  \end{itemize}
\end{frame}

\begin{frame}{Browsing the directory tree}
  \begin{itemize}
  \item Most terminals will start you off in your home directory
  \item Type \texttt{pwd}\ to view your current working directory\\
    (e.g. \textbf{p}rint \textbf{w}orking \textbf{d}irectory)
    \begin{itemize}
    \item \texttt{/} represents the ``root'' of your filesystem
    \item All folders are subfolders of \texttt{/}
    \item Each additional \texttt{/} indicates a new directory
    \end{itemize}
    \pause
  \item Type \texttt{cd ..} to go up a directory
  \item Type in the name of the directory you just left (your home directory)
    to return where you started
  \item Type \texttt{pwd} again to make sure you are in the same place
  \end{itemize}
\end{frame}

\begin{frame}{Browsing the directory tree}
  Next commands to type:
  \begin{itemize}
  \item \texttt{mkdir} makes a new directory
  \item \texttt{touch} makes a new file
  \item \texttt{ls} lists files and directories
  \item \texttt{mv} moves files
  \item \texttt{cp} copies files
  \item \texttt{rm} removes files, \texttt{rm -r} removes folders
  \item \texttt{man} tells you what a command does
  \end{itemize}
  Live demo!
\end{frame}

\begin{frame}{Output redirection/Wildcards}
  \begin{itemize}
  \item Typing \texttt{>} after a command saves its output to a file
  \item Typing an asterisk character (\texttt{*}) as part of a
    file or folder name matches any character string
  \end{itemize}
  Live demo!
\end{frame}

\begin{frame}{Other commonly used applications}
  \begin{itemize}
  \item \texttt{ssh} connects to remote computers
  \item \texttt{tar} is used to compress and uncompress folders
  \item \texttt{for} can be used to loop a command
    (just like for loops in Python)
  \item \texttt{du} is used to find the disk space of files and folders
  \end{itemize}
  Live demo!
\end{frame}

\begin{frame}{Shell activity}
  \begin{itemize}
  \item Run \texttt{make\_mess.py}
    \pause
  \item Remove all folders that have bad\_data in the name
  \item Find which good\_data folders are the largest size and save to
    a file named \texttt{file\_sizes.txt} in your home directory
    \begin{itemize}
    \item Hint: use \texttt{sort} and \texttt{du} with output redirection
    \end{itemize}
  \item Bonus points - move 5 largest folders to a new directory named
    \texttt{best\_data}
  \end{itemize}
\end{frame}

\begin{frame}{Bonus solution}
  \inputminted{bash}{code-snippets/bonus-shell-problem-solution.sh}
\end{frame}