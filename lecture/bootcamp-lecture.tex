\documentclass[xcolor=x11names,compress,aspectratio=43]{beamer}

%% General document Required packages%%%%%%%%%%%%%%%%%%%%%%%%%%%%%%%%%%
\usepackage{graphicx}
\usepackage{grffile}
\usepackage{hanging}% http://ctan.org/pkg/hanging
\usepackage{contour}
\usepackage{color}
%% \usepackage{fontspec}
\usepackage{anyfontsize}
\usepackage{array}
\usepackage{mathtools}
\usepackage{hyperref}
\usepackage{animate}
\usepackage{xcolor}
\usepackage{gensymb}

\usepackage{minted}
\usemintedstyle{borland}

%%%%%%%%%%%%%%%%%%%%%%%%%%%%%%%%%%%%%%%%%%%%%%%%%%%%%%

%% Beamer Layout %%%%%%%%%%%%%%%%%%%%%%%%%%%%%%%%%%
\useoutertheme[subsection=false,shadow]{miniframes}
\useinnertheme{default}
\usefonttheme{default}
%\usepackage{palatino}

%%%% Bibliography
% If you have more than one page of references, you want to tell beamer
% to put the continuation section label from the second slide onwards
\setbeamertemplate{frametitle continuation}[from second]
\setbeamertemplate{bibliography item}[text]{}

%%%%%%%%%%%%%%%%%%%%%%%%%%%
\definecolor{goldenblue}{RGB}{27,27,179}
\definecolor{ucsfteal}{cmyk}{0.43,0.0,0.14,0.21}
\definecolor{ucsfgray}{cmyk}{0.0,0.05,0.10,0.29}
\definecolor{ucsfdarkblue}{cmyk}{1.0,0.55,0.0,0.55}
\definecolor{gold}{cmyk}{0.0,0.18,1.0,0.15}
\definecolor{ucsflightblue}{cmyk}{0.55,0.24,0.0,0.09}

% Hyperref Setup

\hypersetup{
    colorlinks=true,
    linkcolor=black,
    citecolor=black,
    filecolor=black,
    urlcolor=ucsfdarkblue
}

% Font settings
%% \setsansfont{Lato}
%% \setsansfont{Raleway}
%% \defaultfontfeatures{Mapping=tex-text}
%% \usepackage{xunicode}
%% \usepackage{xltxtra}
\setbeamerfont{title like}{shape=\scshape}

%% \setbeamerfont{frametitle}{shape=\scshape}
%% \setbeamerfont{title}{family=\fontspec{Oswald}}
%% \setbeamerfont{frametitle}{family=\fontspec{Oswald}}
%% \setbeamerfont{title}{family=\rm\addfontfeatures{Scale=1.18, Numbers={Lining, Proportional}}}

\setbeamercolor*{lower separation line head}{bg=ucsfteal} 
\setbeamercolor*{normal text}{fg=black,bg=white} 
\setbeamercolor*{alerted text}{fg=red} 
\setbeamercolor*{example text}{fg=black} 
\setbeamercolor*{structure}{fg=black}
\setbeamercolor*{section in sidebar shaded}{fg=ucsfgray} 
 
\setbeamercolor*{palette tertiary}{fg=black,bg=black!10} 
\setbeamercolor*{palette quaternary}{fg=black,bg=black!10} 

% Comment out to add navigation controls
\setbeamertemplate{navigation symbols}{}%remove navigation symbols

\setbeamertemplate{footnote}{%
  \hangpara{2em}{1}%
  \makebox[2em][l]{\insertfootnotemark}\footnotesize\insertfootnotetext\par%
}

\renewcommand{\(}{\begin{columns}}
\renewcommand{\)}{\end{columns}}
\newcommand{\<}[1]{\begin{column}{#1}}
\renewcommand{\>}{\end{column}}
%%%%%%%%%%%%%%%%%%%%%%%%%%%%%%%%%%%%%%%%%%%%%%%%%%

%% Useful commands
\newcommand{\superscript}[1]{\ensuremath{^{\textrm{#1}}}}
\newcommand{\subscript}[1]{\ensuremath{_{\textrm{#1}}}}
\usepackage{comment}

% Blank footnotes
\makeatletter
\def\blfootnote{\xdef\@thefnmark{}\@footnotetext}
\makeatother
\let\oldfootnotesize\footnotesize
\renewcommand*{\footnotesize}{\oldfootnotesize\tiny}

\title[Bootcamp Hacking] % (optional, use only with long paper titles)
{Everything we wish we knew about computers before starting grad school}

%\subtitle
%{} % (optional)

\author[] % (optional, use only with lots of authors)
{Kale Kundert \& Kyle Barlow \\ aka re.compile('K[ya]le')}
% - Use the \inst{?} command only if the authors have different
%   affiliation.

%% \institute[UCSF] % (optional, but mostly needed)
%% {
%%  iPQB\\
%%  University of California, San Francisco
%% % - Use the \inst command only if there are several affiliations.
%% % - Keep it simple, no one is interested in your street address.
%% }

\date[] % (optional)
{2014-09-05}

%\subject{}
% This is only inserted into the PDF information catalog. Can be left
% out. 


\begin{document}

% Delete this, if you do not want the table of contents to pop up at
% the beginning of each subsection:
% Can be set to AtBeginSection or AtBeginSubsection
%\AtBeginSection[]{
%{
%\begin{frame}<beamer>{Outline}
%  \tableofcontents[currentsection,currentsubsection]
%\end{frame}
%}
%}

% If you wish to uncover everything in a step-wise fashion, uncomment
% the following command: 

%\beamerdefaultoverlayspecification{<+->}
\contourlength{2pt} %how thick each copy is
\contournumber{30}  %number of copies

\begin{frame}
 \titlepage
\end{frame}

%\begin{frame}{Outline}
%  \tableofcontents
  % You might wish to add the option [pausesections]
%\end{frame}

\section{Rapid-Fire Knowledge}
\subsection{Rapid-Fire Knowledge}

\begin{frame}{Purpose of this session}
  \begin{itemize}
  \item If you are new to computational science, in the spirit of iPQB,
    we will introduce you to both basic and advanced level tools and
    topics, so that you will have some basic familiarity with them
    if you end up needing to use them later.
    \begin{itemize}
    \item Maybe you'll be inspired to try a computational rotation?
    \item The slides are available online on the bootcamp website
    \end{itemize}
  \item If you are already an experienced programmer, we hope to introduce
    you to some tools and science-specific computer stuff that we wish
    we had known at the beginning of grad school
  \end{itemize}
\end{frame}

\begin{frame}{Programming languages}
  \begin{itemize}
  \item Python
    \begin{itemize}
    \item ``Scientists often need to improvise when trying to interpret results, so they are drawn to dynamic languages which allow them to work very quickly and see results almost immediately'' --- Python creator Guido von Rossum
    \end{itemize}
    \pause
  \item C/C++
    \begin{itemize}
    \item Fast - use when runtime is important
    \end{itemize}
    \pause
  \item Web programming
    \begin{itemize}
    \item Makes your software/results accessible to the world
    \end{itemize}
  \end{itemize}
\end{frame}

\begin{frame}{Text editors}
 \begin{itemize}
  \item \textbf{TextWrangler} Free editor for Mac that should already be 
   installed on your computer.
  \item \textbf{Atom} Free (for the time being) editor for Mac that should 
   already be installed on your computer.
  \item \textbf{Sublime} Not free editor that is not installed on your 
   computer, but people online seem to like it.
  \item \textbf{PyCharm} A Python-focused editor/\href{http://en.wikipedia.org/wiki/Integrated_development_environment}{IDE} 
   that has an open source ``community version''.
  \item \textbf{Vim \& Emacs} Very efficient text editors with steep learning 
   curves.
 \end{itemize}
\end{frame}

\begin{frame}{Backups}
  \begin{itemize}
  \item Back up your laptops!
  \item Once you join a lab, your lab will
    probably provide a solution
  \item \href{http://www.code42.com/crashplan/}{CrashPlan}
    is a good option for personal use,
    and you are eligible for a UCSF discount
  \item Anything automatic works (rsync + tar included)
  \end{itemize}
\end{frame}
\begin{frame}{Computational lab notebooks}
  Disclaimer: This presentation is for informational purposes only and not for the purpose of providing legal advice. 
  \begin{itemize}
  \item Everyone kind of does their own thing, but you should at least do something
  \item Options:
    \begin{itemize}
    \item Edit a wiki page
      (\href{http://openwetware.org/wiki/Main_Page}{open wetware}, 
      your lab's own internal wiki, etc.) -- Wikis aren't just for Wikipedia!
    \item IPython Notebook
    \item Evernote
    \item Google Docs
    \item A text editor
    \end{itemize}

  \item Anything works as long as you write down experiments you did,
    their result, and how to reproduce them (just like wetlab)
  \item You should hear more about this in a formal training
  \end{itemize}
\end{frame}

\begin{frame}{IPython}

 \begin{itemize}

  \item Use \texttt{ipython} instead of \texttt{python} when you want an 
   interactive python interpreter.

  \begin{itemize}
   \item Tab completion
   \item Interactive plotting
   \item Integrated documentation
   \item Shell integration
  \end{itemize}

  \item IPython is easy to use, and there are lots of 
   \href{http://ipython.org/ipython-doc/2/interactive/tutorial.html}{tutorials 
   online}.

 \end{itemize}

\end{frame}

\begin{frame}{IPython Notebook}

 \begin{itemize}

  \item 
   \href{http://ipython.org/ipython-doc/stable/notebook/index.html}{IPython 
   Notebook} provides a Mathematica-like user interface for python.

  \item Notebooks are edited and displayed from a web browser.

  \begin{itemize}
   \item Plots immediately show up in-line with your code.
   \item Can use LaTeX to describe your equations.
   \item Easy to distribute.
  \end{itemize}

 \end{itemize}

\end{frame}

\begin{frame}{Stack Overflow}

 \begin{itemize}

  \item Looking up things you don't know is an important part of being a 
   programmer.

  \item \href{http://stackoverflow.com/}{Stack Overflow} is an important 
   resource because it has good answers to a lot of common programming 
   question.

 \end{itemize}

\end{frame}

\begin{frame}{Scientific Computing}

 The following python packages are incredibly useful for scientific computing:

 \begin{itemize}

  \item \href{http://wiki.scipy.org/Tentative_NumPy_Tutorial}{\texttt{numpy}} 
   -- Really fast array processing.

  \item 
   \href{http://docs.scipy.org/doc/scipy/reference/tutorial/general.html}{\texttt{scipy}}
    -- Lots of useful scientific algorithms.

   \begin{itemize}
    \item Linear algebra, clustering, optimization, signal processing, 
     statistics, and more!
   \end{itemize}

  \item 
   \href{http://matplotlib.org/users/pyplot_tutorial.html}{\texttt{matplotlib}} 
   -- Plotting library.

  \item 
   \href{http://pandas.pydata.org/pandas-docs/stable/10min.html}{\texttt{pandas}} 
   -- Easily interact with tabular data.

 \end{itemize}

\end{frame}

\begin{frame}{Open Source Software}

 \begin{itemize}

  \item Open source software isn't just software you don't have to pay for.  
   It's software that anyone has the right to browse, modify, and distribute. 

  \item When you write a piece of software, you should give it a license so 
   that other people know whether or not they can use it.  It's usually not 
   hard to \href{http://choosealicense.com/}{choose a license} you 
   like\footnote{License summaries paraphrased from 
   \url{http://choosealicense.com}}:

  \begin{itemize}

   \item MIT License: Very few restrictions on how your code can be used.

   \item Apache License: Similar to MIT, but gives more protection to your 
    users if you're planning to patent your code.

   \item GPL License: Requires others who use your code to also use an open 
    source license.

   \end{itemize}
   
  \item Legal disclaimer: You should probably check with your PI before 
   choosing a license.

 \end{itemize}

\end{frame}


\section{Shell Activity}
\subsection{Shell Activity}

\begin{frame}{BFF: The Command Line And You}
  Why use the command line?
  \begin{itemize}
  \item Using the command line is often faster than using an equivalent GUI
    (\textbf{G}raphical \textbf{U}ser \textbf{I}nterface) tool
  \item Many computing operations are only possible via the command line
  \item The command line is used to access and control remote servers,
    including scientific computing clusters
  \end{itemize}
\end{frame}

\begin{frame}{The terminal}
  \begin{itemize}
  \item The application you use to interact with the command line is called a
    \textit{terminal}
  \item On OSX, the terminal can be accessed by going to
    Applications $\rightarrow$ Utilities $\rightarrow$ Terminal
  \item Fire one up now!
  \end{itemize}
\end{frame}

\begin{frame}{Browsing the directory tree}
  \begin{itemize}
  \item Most terminals will start you off in your home directory
  \item Type \texttt{pwd}\ to view your current working directory\\
    (e.g. \textbf{p}rint \textbf{w}orking \textbf{d}irectory)
    \begin{itemize}
    \item \texttt{/} represents the ``root'' of your filesystem
    \item All folders are subfolders of \texttt{/}
    \item Each additional \texttt{/} indicates a new directory
    \end{itemize}
    \pause
  \item Type \texttt{cd ..} to go up a directory
  \item Type in the name of the directory you just left (your home directory)
    to return where you started
  \item Type \texttt{pwd} again to make sure you are in the same place
  \end{itemize}
\end{frame}

\begin{frame}{Browsing the directory tree}
  Next commands to type:
  \begin{itemize}
  \item \texttt{mkdir} makes a new directory
  \item \texttt{touch} makes a new file
  \item \texttt{ls} lists files and directories
  \item \texttt{mv} moves files
  \item \texttt{cp} copies files
  \item \texttt{rm} removes files, \texttt{rm -r} removes folders
  \item \texttt{man} tells you what a command does
  \end{itemize}
  Live demo!
\end{frame}

\begin{frame}{Output redirection/Wildcards}
  \begin{itemize}
  \item Typing \texttt{>} after a command saves its output to a file
  \item Typing an asterisk character (\texttt{*}) as part of a
    file or folder name matches any character string
  \end{itemize}
  Live demo!
\end{frame}

\begin{frame}{Other commonly used applications}
  \begin{itemize}
  \item \texttt{ssh} connects to remote computers
  \item \texttt{tar} is used to compress and uncompress folders
  \item \texttt{for} can be used to loop a command
    (just like for loops in Python)
  \item \texttt{du} is used to find the disk space of files and folders
  \item \texttt{grep} can pull out specific lines of text
  \end{itemize}
  Live demo!
\end{frame}

\begin{frame}{Shell activity}
  \begin{itemize}
  \item Run \texttt{make\_mess.py} - download from iPQB website
    \pause
  \item Remove all folders that have bad\_data in the name
  \item Find which good\_data folders are the largest size and save to
    a file named \texttt{file\_sizes.txt} in your home directory
    \begin{itemize}
    \item Hint: use \texttt{sort} and \texttt{du} with output redirection
    \end{itemize}
  \item Find all total\_score lines
    \begin{itemize}
    \item Bonus: remove files that are missing total\_score lines
      (they will have 'SEGFAULT' at the end instead)
    \end{itemize}
  \item Bonus points - move 5 largest folders to a new directory named
    \texttt{best\_data}
  \end{itemize}
\end{frame}

\begin{frame}{Bonus solution}
  \inputminted{bash}{code-snippets/bonus-shell-problem-solution.sh}
\end{frame}

\section{Git/Python Activity}
\subsection{Git/Python Activity}

%% \begin{frame}{Command-line Essentials}

 \begin{itemize}

  \item \texttt{ls} -- List all the files in the current directory.

  \item \texttt{cd} -- Change to a new directory.

  \begin{itemize}
   \item Use \texttt{cd ..} to move up the directory hierarchy.
   \item Use \texttt{cd -} to go back to the last directory you were in.
  \end{itemize}

  \item \texttt{mv source target} -- Move (i.e. rename) a file from 
   \texttt{source} to \texttt{target}.

  \item \texttt{cp source target} -- Copy a file from \texttt{source} to 
   \texttt{target}.

  \item \texttt{rm target} -- Remove (i.e. delete) the \texttt{target} file.

  \item \texttt{man command} -- Get help on how to use the given 
   \texttt{command}.

 \end{itemize}

\end{frame}


\begin{frame}{Why Version Control?}

 \begin{itemize}

  \item The purpose of version control is to track changes you make to your 
   code.

  \item This makes it easy to jump back to old versions of your code.

  \begin{itemize}
   \item So you can make changes fearlessly!
  \end{itemize}

  \item This also makes it easy to share changes with other people.

  \item It's a good idea to always use version control.
 
 \end{itemize}

\end{frame}



\begin{frame}{Creating a repository}

 There are two ways to create a repository:

 \begin{itemize}

  \item \texttt{git init} -- Turns the current directory into a git repository.  
   Do this when you start a new project.

  \item \texttt{git clone} -- Copy a repository onto your computer from 
   somewhere else.

  \item \href{https://github.com/}{GitHub} is a popular website that hosts 
   public git repositories for free.

  \item The material we'll be using for this activity is stored in a repository 
   on GitHub, so our first step will be to clone it:

   {\fontsize{9}{10}\selectfont
   \mint{bash}|git clone https://github.com/ipqb/bootcamp-primes-activity.git|
   }

 \end{itemize}

\end{frame}

\begin{frame}{Committing Your Changes}

 \begin{itemize}

  \item You have to tell git which changes you want it to keep track of.

  \item This process is called committing, and in git it has two steps:

  \item \texttt{git add} -- Add files to a staging area.

  \item \texttt{git commit} -- Commit all the files in the staging area.

 \end{itemize}
 
\end{frame}

\begin{frame}{Sharing Your Code}

 \begin{itemize}

  \item Git makes it easy to share your code and to develop it in concert with 
   other people.

  \item \texttt{git push} -- Push all of your commits to a central server.

  \item \texttt{git pull} -- Pull any commits made by others into your 
   repository.

 \end{itemize}

\end{frame}

\begin{frame}{Activity: Comparing Prime-Finding Algorithms}

 \begin{itemize}
   
  \item This activity is meant to give some some experience with git and a 
   little more practice with python.

  \item I've written two algorithms to find all the prime numbers below some 
  threshold.  The code is in \texttt{find\_primes.py}.

  \item Spend a few minutes reading the two algorithms, understanding how they 
   work, and thinking about which is more efficient and why.

 \end{itemize}

\end{frame}

\begin{frame}{Optimizing Code}

 \begin{itemize}

  \item Optimizing code is all about cutting out redundant work being done by 
   the computer.

  \item A common way to optimize python code is to convert explicit loops into 
   implied ones.  This is what makes \texttt{numpy} so powerful.

  \item Don't optimize code unless you have to.  There's no point optimizing 
   code that isn't a bottleneck.  Doing so will probably just make your code 
   harder to understand.

  \item In addition to thinking about efficient use of the CPU, we can also 
   think about efficient use of memory.

 \end{itemize}

\end{frame}

\begin{frame}{Activity: Comparing Prime-Finding Algorithms}

 Work on the following activity until 5:00 (then we'll get burritos!)

 \begin{itemize}

  \item Make a plot comparing how long each function takes as the input limit 
   gets higher and higher (i.e. as more and more primes are found).

  \begin{itemize}

   \item Put your code in \texttt{time\_primes.py}.

   \item Use \texttt{matplotlib} to create the plot.

  \end{itemize}

  \item The trial division function can be made more efficient.  If you can 
   think of an optimization, implement it and see how much faster it makes the 
   algorithm.

  \item Commit your work at least once!  My rule of thumb is to commit every 
   time I get something new working.

 \end{itemize}

\end{frame}



\section{Rama Project}
\subsection{Rama Project}

\begin{frame}{Download my solutions}

 \begin{figure}[h]
  {\large \url{https://github.com/kalekundert/rama-project.git}}
 \end{figure}

 Two solutions:

 \begin{itemize}

  \item \texttt{simple-rama.py} -- A simple solution written to be as easy to 
   understand as possible.

  \item \texttt{fancy-rama.py} -- A complete solution written to be an example 
   of idiomatic python code.
  
 \end{itemize}

\end{frame}

\begin{frame}{Code Review}

 \begin{itemize}

  \item For the rest of the afternoon, we're going to do a code review.

  \item The purpose of a code review is to get another person to look at your 
   code and give feedback.  Two pairs of eyes are better than one!

  \item Everyone has been assigned a partner.  Everyone should explain their 
   code to their partner and give feedback about their partner's code.  If you 
   approached the problem in different ways, talk about it and decide which 
   approach was better.

  \item Be mindful of giving constructive criticism!

 \end{itemize}

\end{frame}



\end{document}
