\documentclass[xcolor=x11names,compress,aspectratio=43]{beamer}

%% General document Required packages%%%%%%%%%%%%%%%%%%%%%%%%%%%%%%%%%%
\usepackage{graphicx}
\usepackage{grffile}
\usepackage{hanging}% http://ctan.org/pkg/hanging
\usepackage{contour}
\usepackage{color}
%% \usepackage{fontspec}
\usepackage{anyfontsize}
\usepackage{array}
\usepackage{mathtools}
\usepackage{hyperref}
\usepackage{animate}
\usepackage{xcolor}
\usepackage{gensymb}

\usepackage{minted}
\usemintedstyle{borland}

%%%%%%%%%%%%%%%%%%%%%%%%%%%%%%%%%%%%%%%%%%%%%%%%%%%%%%

%% Beamer Layout %%%%%%%%%%%%%%%%%%%%%%%%%%%%%%%%%%
\useoutertheme[subsection=false,shadow]{miniframes}
\useinnertheme{default}
\usefonttheme{default}
%\usepackage{palatino}

%%%% Bibliography
% If you have more than one page of references, you want to tell beamer
% to put the continuation section label from the second slide onwards
\setbeamertemplate{frametitle continuation}[from second]
\setbeamertemplate{bibliography item}[text]{}

%%%%%%%%%%%%%%%%%%%%%%%%%%%
\definecolor{goldenblue}{RGB}{27,27,179}
\definecolor{ucsfteal}{cmyk}{0.43,0.0,0.14,0.21}
\definecolor{ucsfgray}{cmyk}{0.0,0.05,0.10,0.29}
\definecolor{ucsfdarkblue}{cmyk}{1.0,0.55,0.0,0.55}
\definecolor{gold}{cmyk}{0.0,0.18,1.0,0.15}
\definecolor{ucsflightblue}{cmyk}{0.55,0.24,0.0,0.09}

% Hyperref Setup

\hypersetup{
    colorlinks=true,
    linkcolor=black,
    citecolor=black,
    filecolor=black,
    urlcolor=ucsfdarkblue
}



% Font settings
%% \setsansfont{Lato}
%% \setsansfont{Raleway}
%% \defaultfontfeatures{Mapping=tex-text}
%% \usepackage{xunicode}
%% \usepackage{xltxtra}
\setbeamerfont{title like}{shape=\scshape}

%% \setbeamerfont{frametitle}{shape=\scshape}
%% \setbeamerfont{title}{family=\fontspec{Oswald}}
%% \setbeamerfont{frametitle}{family=\fontspec{Oswald}}
%% \setbeamerfont{title}{family=\rm\addfontfeatures{Scale=1.18, Numbers={Lining, Proportional}}}

\setbeamercolor*{lower separation line head}{bg=ucsfteal} 
\setbeamercolor*{normal text}{fg=black,bg=white} 
\setbeamercolor*{alerted text}{fg=red} 
\setbeamercolor*{example text}{fg=black} 
\setbeamercolor*{structure}{fg=black}
\setbeamercolor*{section in sidebar shaded}{fg=ucsfgray} 
 
\setbeamercolor*{palette tertiary}{fg=black,bg=black!10} 
\setbeamercolor*{palette quaternary}{fg=black,bg=black!10} 

% Comment out to add navigation controls
\setbeamertemplate{navigation symbols}{}%remove navigation symbols

\setbeamertemplate{footnote}{%
  \hangpara{2em}{1}%
  \makebox[2em][l]{\insertfootnotemark}\footnotesize\insertfootnotetext\par%
}

\renewcommand{\(}{\begin{columns}}
\renewcommand{\)}{\end{columns}}
\newcommand{\<}[1]{\begin{column}{#1}}
\renewcommand{\>}{\end{column}}
%%%%%%%%%%%%%%%%%%%%%%%%%%%%%%%%%%%%%%%%%%%%%%%%%%

%% Useful commands
\newcommand{\superscript}[1]{\ensuremath{^{\textrm{#1}}}}
\newcommand{\subscript}[1]{\ensuremath{_{\textrm{#1}}}}
\usepackage{comment}

% Blank footnotes
\makeatletter
\def\blfootnote{\xdef\@thefnmark{}\@footnotetext}
\makeatother
\let\oldfootnotesize\footnotesize
\renewcommand*{\footnotesize}{\oldfootnotesize\tiny}

\title[Bootcamp Hacking] % (optional, use only with long paper titles)
{Everything we wish we knew about computers before starting grad school}

%\subtitle
%{} % (optional)

\author[] % (optional, use only with lots of authors)
{Kale Kundert \& Kyle Barlow}
% - Use the \inst{?} command only if the authors have different
%   affiliation.

%% \institute[UCSF] % (optional, but mostly needed)
%% {
%%  iPQB\\
%%  University of California, San Francisco
%% % - Use the \inst command only if there are several affiliations.
%% % - Keep it simple, no one is interested in your street address.
%% }

\date[] % (optional)
{2014-09-05}

%\subject{}
% This is only inserted into the PDF information catalog. Can be left
% out. 


\begin{document}

% Delete this, if you do not want the table of contents to pop up at
% the beginning of each subsection:
% Can be set to AtBeginSection or AtBeginSubsection
%\AtBeginSection[]{
%{
%\begin{frame}<beamer>{Outline}
%  \tableofcontents[currentsection,currentsubsection]
%\end{frame}
%}
%}

% If you wish to uncover everything in a step-wise fashion, uncomment
% the following command: 

%\beamerdefaultoverlayspecification{<+->}
\contourlength{2pt} %how thick each copy is
\contournumber{30}  %number of copies

\begin{frame}
 \titlepage
\end{frame}

%\begin{frame}{Outline}
%  \tableofcontents
  % You might wish to add the option [pausesections]
%\end{frame}

\section{Rapid-Fire Knowledge}
\subsection{Rapid-Fire Knowledge}

\begin{frame}{Purpose of this session}
  \begin{itemize}
  \item If you are new to computational science, in the spirit of iPQB,
    we will introduce you to both basic and advanced level tools and
    topics, so that you will have some basic familiarity with them
    if you end up needing to use them later.
    \begin{itemize}
    \item Maybe you'll be inspired to try a computational rotation?
    \item The slides are available online on the bootcamp website
    \end{itemize}
  \item If you are already an experienced programmer, we hope to introduce
    you to some tools and science-specific computer stuff that we wish
    we had known at the beginning of grad school
  \end{itemize}
\end{frame}

\begin{frame}{Programming languages}
  \begin{itemize}
  \item Python
    \begin{itemize}
    \item ``Scientists often need to improvise when trying to interpret results, so they are drawn to dynamic languages which allow them to work very quickly and see results almost immediately'' --- Python creator Guido von Rossum
    \end{itemize}
    \pause
  \item C/C++
    \begin{itemize}
    \item Fast - use when runtime is important
    \end{itemize}
    \pause
  \item Web programming
    \begin{itemize}
    \item Makes your software/results accessible to the world
    \end{itemize}
  \end{itemize}
\end{frame}

\begin{frame}{IPython}

 \begin{itemize}

  \item Use \texttt{ipython} instead of \texttt{python} when you want an 
   interactive python interpreter.

  \begin{itemize}
   \item Tab completion
   \item Interactive plotting
   \item Integrated documentation
   \item Shell integration
  \end{itemize}

  \item IPython is easy to use, and there are lots of 
   \href{http://ipython.org/ipython-doc/2/interactive/tutorial.html}{tutorials 
   online}.

 \end{itemize}

\end{frame}

\begin{frame}{IPython Notebook}

 \begin{itemize}

  \item 
   \href{http://ipython.org/ipython-doc/stable/notebook/index.html}{IPython 
   Notebook} provides a Mathematica-like user interface for python.

  \item Notebooks are edited and displayed from a web browser.

  \begin{itemize}
   \item Plots immediately show up in-line with your code.
   \item Can use LaTeX to describe your equations.
   \item Easy to distribute.
  \end{itemize}

 \end{itemize}

\end{frame}


\section{Git Activity}
\subsection{Git Activity}

\begin{frame}{This is a slide with code}
 \begin{itemize}
 \item Item 1
 \item Item 2
 \end{itemize}
 \inputminted{bash}{example-script.sh}
\end{frame}


\section{Shell Activity}
\subsection{Shell Activity}

\begin{frame}{This is a slide with code}
 \begin{itemize}
 \item Item 1
 \item Item 2
 \end{itemize}
 \inputminted{bash}{example-script.sh}
\end{frame}


\end{document}
